\usepackage{parskip}
\parskip=2\baselineskip \advance\parskip by 0pt plus 1pt

\chapter{Introduction}
\label{introchap}From protists to humans, all plants and animals 
live in close association with microbial organisms. 
The microbiome is the full collection of these microbes, 
their genome, and their environmental interactions. 
Scientists are only now starting to appreciate the complex interactions between
microbial communities and organisms. It is important to 
study these complex interactions because they have been 
linked with initiating ailments, as well as fostering health. 
In particular, recent studies have reveled that microbial communities
have been associated with diseases such as diabetes, obesity, 
and rheumatoid arthritis.  Furthermore, microbiomes are also 
being studied in the context of the environment as well. 
With the amount of recent interest in studying microbiomes, 
some consider it to be a “newly discovered organ”.

\includegraphics[scale=0.44]{Rplot.pdf}


Although modern technology such as DNA testing has 
allowed scientists to start exploring microbiomes, 
the methods in processing and understanding microbial 
data are still evolving. In other words, although scientists 
are often able to collect a lot of data, they often have a 
difficult time interpreting the data. In this thesis we discuss 
a machine learning principle called rule induction which yields 
insight into microbial data and help understand the complex 
interactions that occur between organisms.


This sample document illustrates how to use the
{\tt thesis} class, originally written by John P. Weiss.
Some requirements of the Graduate School are written
into that file; page size, line spacing, appropriate
placement of captions for tables and figures, etc.
Other tasks of conforming to the requirements are
left to other existing \LaTeX{} packages.
For example, a common problem is to insert graphics ---
figures and tables --- into the body of the thesis.  For
this one should use the {\tt graphicx} package, which is
part of the standard \TeX{} distribution.  Likewise, the
Grad School specs say that a large table may be displayed
in landscape mode at reduced size, but its caption must
also be in rotated position, in the same font and size as
the normal text in the body of the thesis.  To accomplish
this, the user must invoke the {\tt rotating} package,
available online.


Figure \ref{xfigDiagram} shows something or other;
the image is from a PDF file imported into this document
using the \verb2graphicx2 package.
The command \verb2\usepackage{graphicx}2, which appears
near the very top of the main \LaTeX{} file, reads in
this package which defines the
\verb2\includegraphics{}2 macro.


\begin{figure}[htbp] \label{xfigDiagram}
    \caption[Cylinder and measurements]{
	This diagram of a cylinder and various
	measurements and quantities was actually
	made using {\bf xfig}, a freeware
	drawing program for Unix systems.
	Diagrams can be exported directly to PDF
	files, the preferred format for
	vector graphics.  Vector graphics can
	be magnified indefinitely without degradation,
	whereas bitmap images (JPG and PNG)
	must be pretty high-resolution if you don't
	want them looking all pixellated when
	magnified.
	}
    \begin{center}
	\includegraphics[width=100mm]{figs/cyl.pdf}
    \end{center}
\end{figure}

\section{Lists in {\tt thesis} class}

In {\tt thesis} class (for Colorado University),
lists are defined so that nested lists will be
numbered or marked appropriately.
First, an itemized (non-enumerated) list
prefaces each item with a bullet.
Nested itemized list use asterisks,
then dashes, then dots.
These lists are typed between
the \verb2\begin{itemize}2
and \verb2\end{itemize}2
commands.

\begin{itemize}
  \item{} This is ``itemized'' item A.
  \item{} This is ``itemized'' item B.
  \item{} This is ``itemized'' item C.
  \begin{itemize}
    \item{} This is ``itemized'' subitem A.
    \begin{itemize}
      \item{} This is ``itemized'' subsubitem A.
      \begin{itemize}
        \item{} This is ``itemized'' subsubsubitem A.
      \end{itemize}
      \item{} This is ``itemized'' subsubitem B.
    \end{itemize}
    \item{} This is ``itemized'' subitem B.
  \end{itemize}
  \item{} This is ``itemized'' item D.
\end{itemize}

Enumerated lists use the commands
\verb2\begin{enumerate}2 and
\verb2\end{enumerate}2,
and nested enumerations appear like this.

\begin{enumerate}
  \item{} This is ``enumerated'' item A.
  \item{} This is ``enumerated'' item B.
  \item{} This is ``enumerated'' item C.
  \begin{enumerate}
    \item{} This is ``enumerated'' subitem A.
    \begin{enumerate}
      \item{} This is ``enumerated'' subsubitem A.
      \begin{enumerate}
        \item{} This is ``enumerated'' subsubsubitem A.
      \end{enumerate}
      \item{} This is ``enumerated'' subsubitem B.
    \end{enumerate}
    \item{} This is ``enumerated'' subitem B.
  \end{enumerate}
  \item{} This is ``enumerated'' item D.
\end{enumerate}


The work presented
here\footnote{Footnotes are handled neatly by \LaTeX.}
is an extension of Taum\cite{taum}
and Lao et~al.\cite{lao:paper},
fictional references that are in the bibliographic
source file \refs.bib.

\begin{table}[htb] \label{powertable}
    \caption[Example of a table with its own footnotes]{
	Here is an example of a table with its own footnotes.
	Don't use the $\backslash${\tt footnote} macro if you
	don't want the footnotes at the bottom of the page.
	Also, note that in a thesis the caption goes
	\emph{above} a table, unlike figures.
	}
    \begin{center}
    \begin{tabular}{||l|c|c|c|c||} \hline
	& $S$ & $P$ &   $Q^{\ast}$  & $D^{\dagger}$ \\	% footnote symbols!
	wave form & (kVA) & (kW) & (kVAr) & (kVAd) \\  \hline \hline
	Fig.  \ref{xfigDiagram}a  & 25.48 & 25.00 & -2.82 & 4.03 \\ \hline
	Fig.  \ref{xfigDiagram}b  & 25.11 & 18.02 & -9.75 & 14.52 \\ \hline
	Table \ref{pdftable}  & 24.98 & 22.26 & 9.19 & 6.64 \\ \hline
	Table \ref{powertable}  & 23.48 & 15.00 & 6.59 & 16.82 \\ \hline
	Fig.  \ref{pyramid}  & 24.64 & 22.81 & -0.44 & 9.3 \\ \hline
	\end{tabular}
   \\ \rule{0mm}{5mm}
   ${}^\ast$kVAr means reactive power.		% footnote symbol
\\ ${}^\dagger$kVAd means distortion power.	% footnote symbol
\end{center}
\end{table}


